
%%%%%%%%%%%%%%%%%%%%%%%%%%%%%%%%%%%%%%%%%%%%%%%%%%%%%%%%%%%%%%%%%%%%%
%% This is a (brief) model paper using the achemso class
%% The document class accepts keyval options, which should include
%% the target journal and optionally the manuscript type.
%%%%%%%%%%%%%%%%%%%%%%%%%%%%%%%%%%%%%%%%%%%%%%%%%%%%%%%%%%%%%%%%%%%%%
\documentclass[journal=jacsat,manuscript=article]{achemso}

%%%%%%%%%%%%%%%%%%%%%%%%%%%%%%%%%%%%%%%%%%%%%%%%%%%%%%%%%%%%%%%%%%%%%
%% Place any additional packages needed here.  Only include packages
%% which are essential, to avoid problems later.
%%%%%%%%%%%%%%%%%%%%%%%%%%%%%%%%%%%%%%%%%%%%%%%%%%%%%%%%%%%%%%%%%%%%%
\usepackage{chemformula} % Formula subscripts using \ch{}
\usepackage[T1]{fontenc} % Use modern font encodings
\usepackage[outdir=./Figures/]{epstopdf}

%%%%%%%%%%%%%%%%%%%%%%%%%%%%%%%%%%%%%%%%%%%%%%%%%%%%%%%%%%%%%%%%%%%%%
%% If issues arise when submitting your manuscript, you may want to
%% un-comment the next line.  This provides information on the
%% version of every file you have used.
%%%%%%%%%%%%%%%%%%%%%%%%%%%%%%%%%%%%%%%%%%%%%%%%%%%%%%%%%%%%%%%%%%%%%
%%\listfiles

%%%%%%%%%%%%%%%%%%%%%%%%%%%%%%%%%%%%%%%%%%%%%%%%%%%%%%%%%%%%%%%%%%%%%
%% Place any additional macros here.  Please use \newcommand* where
%% possible, and avoid layout-changing macros (which are not used
%% when typesetting).
%%%%%%%%%%%%%%%%%%%%%%%%%%%%%%%%%%%%%%%%%%%%%%%%%%%%%%%%%%%%%%%%%%%%%
\newcommand*\mycommand[1]{\texttt{\emph{#1}}}

%%%%%%%%%%%%%%%%%%%%%%%%%%%%%%%%%%%%%%%%%%%%%%%%%%%%%%%%%%%%%%%%%%%%%
%% Meta-data block
%% ---------------
%% Each author should be given as a separate \author command.
%%
%% Corresponding authors should have an e-mail given after the author
%% name as an \email command. Phone and fax numbers can be given
%% using \phone and \fax, respectively; this information is optional.
%%
%% The affiliation of authors is given after the authors; each
%% \affiliation command applies to all preceding authors not already
%% assigned an affiliation.
%%
%% The affiliation takes an option argument for the short name.  This
%% will typically be something like "University of Somewhere".
%%
%% The \altaffiliation macro should be used for new address, etc.
%% On the other hand, \alsoaffiliation is used on a per author basis
%% when authors are associated with multiple institutions.
%%%%%%%%%%%%%%%%%%%%%%%%%%%%%%%%%%%%%%%%%%%%%%%%%%%%%%%%%%%%%%%%%%%%%
\author{Daniel Hendrickson}
\email{dhendrickson@portofvirginia.com}
\email{dchendrickson01@email.wm.edu}
\phone{+1 (202) 374-0651}
\affiliation[Port of Virginia]
{Vice President Asset Management, Virginia Port Authority, Norfolk, VA}
\alsoaffiliation[The College of William and Mary]
{Department of Applied Science, William and Mary, Williamsburg, VA}


%%%%%%%%%%%%%%%%%%%%%%%%%%%%%%%%%%%%%%%%%%%%%%%%%%%%%%%%%%%%%%%%%%%%%
%% The document title should be given as usual. Some journals require
%% a running title from the author: this should be supplied as an
%% optional argument to \title.
%%%%%%%%%%%%%%%%%%%%%%%%%%%%%%%%%%%%%%%%%%%%%%%%%%%%%%%%%%%%%%%%%%%%%
\title[Research Prospecutus]
  {Plan for research on Rail Mounted Gantry Crane Structural Health Monitoring and Non Destructive Evaluation}

%%%%%%%%%%%%%%%%%%%%%%%%%%%%%%%%%%%%%%%%%%%%%%%%%%%%%%%%%%%%%%%%%%%%%
%% Some journals require a list of abbreviations or keywords to be
%% supplied. These should be set up here, and will be printed after
%% the title and author information, if needed.
%%%%%%%%%%%%%%%%%%%%%%%%%%%%%%%%%%%%%%%%%%%%%%%%%%%%%%%%%%%%%%%%%%%%%
\abbreviations{IR,NMR,UV}
\keywords{Helmholtz, Waves, Scatter}


%%%%%%%%%%%%%%%%%%%%%%%%%%%%%%%%%%%%%%%%%%%%%%%%%%%%%%%%%%%%%%%%%%%%%
%% The manuscript does not need to include \maketitle, which is
%% executed automatically.
%%%%%%%%%%%%%%%%%%%%%%%%%%%%%%%%%%%%%%%%%%%%%%%%%%%%%%%%%%%%%%%%%%%%%
\begin{document}

%%%%%%%%%%%%%%%%%%%%%%%%%%%%%%%%%%%%%%%%%%%%%%%%%%%%%%%%%%%%%%%%%%%%%
%% The "tocentry" environment can be used to create an entry for the
%% graphical table of contents. It is given here as some journals
%% require that it is printed as part of the abstract page. It will
%% be automatically moved as appropriate.
%%%%%%%%%%%%%%%%%%%%%%%%%%%%%%%%%%%%%%%%%%%%%%%%%%%%%%%%%%%%%%%%%%%%%
\begin{tocentry}

\includegraphics{Figures/PortLogo}

\end{tocentry}

%%%%%%%%%%%%%%%%%%%%%%%%%%%%%%%%%%%%%%%%%%%%%%%%%%%%%%%%%%%%%%%%%%%%%
%% The abstract environment will automatically gobble the contents
%% if an abstract is not used by the target journal.
%%%%%%%%%%%%%%%%%%%%%%%%%%%%%%%%%%%%%%%%%%%%%%%%%%%%%%%%%%%%%%%%%%%%%
\begin{abstract}
 
Do the Prospectus thing so I can get to the next level at school
  

\end{abstract}
\pagebreak
%%%%%%%%%%%%%%%%%%%%%%%%%%%%%%%%%%%%%%%%%%%%%%%%%%%%%%%%%%%%%%%%%%%%%
%% Start the main part of the manuscript here.
%%%%%%%%%%%%%%%%%%%%%%%%%%%%%%%%%%%%%%%%%%%%%%%%%%%%%%%%%%%%%%%%%%%%%



\pagebreak
\section{Chapter 1: Introduction, Statement of Problem}
The Port of Virginia is the 5th largest in the US, 3rd largest on East Coast. 2nd Largest single operator in the US. In 2008 APM-T went live with a rail mounted gantry crane terminal in Portsmouth. Virginia Port Authority took over running it in 2010. That facility had 30 Rail Mounted Auto Stacking Cranes (ASC) in fifteen stacks. These cranes are now 12 years into their expected 22 year life span, and are starting to see wear. 
\begin{figure}
	\centering
	\includegraphics[width=0.7\linewidth]{"Figures/VIG Overview"}
	\caption[VIG Terminal Areal Image]{The Virginia International Gateway terminal in Portsmouth, Virginia is capable of handling 1.2 million containers per year.  The 28 stacks of containers in the yard are managed by 56 Automatic Stacking Cranes (ASC) that work in a semi-automated manner.  This image is rotated by -90 degrees, to fit on the page.  The top of the image is the water front, where the berth is capable of holding 3 neo-post-panamax container ships and loading and unloading them with 12 ship to shore cranes, 4 of which were the largest in the Western Hemisphere when installed, and remain the largest on the East Coast of North America.}
	\label{fig:VIG_overhead}
\end{figure}
The operational model pioneered by the facility, now called Virginia International Gateway (VIG) was so successful that the port replicated the system at Norfolk International Terminal (NIT) adding an addition 60 ASCs in 30 stacks. Also, VIG was expanded with an additional 13 stacks with 26 ASCs. At the time the expansion was announced, the 86 ASCs ordered for VIG and NIT was the largest single order for cranes in the port industry history. These two projects represented investments of \$460m and \$320m respectively.  The last NIT new ASC came online in December of 2020. 

Now in full operation, and with 10 months of data, the fleet of 86 new and 30 original ASCs will cost the port over \$17.5m in maintenance expenses. There is additional costs to team operating the terminal that can be significant. While a ship is working, roughly 3 stacks are supporting each crane, and the containers are needed in a specific order. If an ASC were to break during the operation, a the Quay crane with 26 workers has to have to wait for a repair. This costs in order of \$2,500 an hour during the day, and \$3,750 an hour nights and weekends.


\pagebreak
\section{Chapter 2: Literature Review}
\subsection{Electrodynamics Finite Integration Technique}
Electrodynamics Finite Integration Technique (EFIT) is a methodology for simulating elastic and acoustic waves as they pass through medium.  It uses a discretized grid of points representing small cubes on which the stresses and motion the elastic waves work.  The elastic waves are given by the field equations:
\begin{equation}
	\rho \frac{\partial v_i}{\partial t} - \frac{\partial}{\partial i}\sigma_{ij} = F_i
\end{equation}
Where $\rho$ is the initial at rest density of the material.  Density times time derivative of velocity is clearly the equivalent to $F=ma$ for the unit volume.  $\sigma_{ij}$ is the stress on the i dimensional face in the j direction.  When i = j it is the normal stress, and when i =\= j it is the shear stresses.  F is the force applied in the i dimension. 

The grid for the Stresses and the Velocities can be set up with the centers on each other’s vertices of the cubes.  This allows the current velocity to be calculated by the change in velocity, which is calculated by the stresses a the previous half time step.
In turn, the stresses can be calculated by the form stresses, plus the change in stresses calculated by the velocities at the last half time step.
In this fashion, the whole time series can be calculated one by half stepping through time, calculating the stresses and then velocities in turn.
Initially proposed in Fellinger \cite{fellinger_numerical_1995} in his dissertation in 1991.  He was a student at the University of Kassel, Department of Electrical Engineering/Computer Science.  The work was inspired by the dual grid integration half step through time developed for Electromagnetic by Yee\cite{kane_yee_numerical_1966}.  It was also built off of Maradiaga \cite{madariaga_dynamics_1976} who first used a dual staggered grid system for electrodynamics in two dimensions, but with a finite difference method, as opposed to finite integration.  Marklien, a subsequent student in the University of Kassel, wrote a chapter on EM FIT, EFIT, and AFIT\cite{marklein_11_nodate}, laying out the three methods, and the similarities in the algorithm’s and techniques.  
Bingham in 2008\cite{bingham_ultrasonic_2008} wrote EFIT code to run in a highly paralyzed environment.  She also was able to show the ability to make complex shapes with shared features that could run efficiently, and be displayed clearly.  The data for elastic waves running through aircraft structural stringers closely matched experimental results.  Bingham did follow on work on more efficient parallel algorithms for EFIT in
Lecky\cite{leckey_multiple-mode_2012} followed on Bingham’s work and created 3D ultra sound simulations and modeled Lamb waves and their multiple modes.  The Lamb wave simulation via EFIT was able to show the expected separation between multiple modes.  It also matched experimental data where a metal plate was intentionally damaged.  The model then was able to be used to represent other damage systems.


\subsection{Rail Structural Health Monitoring}
Largely currently conducted using ultra sound. Historically needed a whole train car full of racks for computers, now can be done on a small human portable rolling cane.

Found an article that summarize the current state of ultrasonics in rail heal monitoring, published earlier this year by Bombarda\cite{bombarda_rail_2021}. The paper is focused on Ultrasonic methods and does not include the Eddy Current methods.
Next I am reading one of his cites, a state of the field written in 2014 by Rizzo\cite{rizzo_sensing_2014}.  


\subsection{Ultrasonic Examination in Non-Destructive Evaluation}
Raleigh waves, crack detection lamb waves in plates 
Started with Rose 2002
•         Found Felice 2018 a review of 2000-2018.
•         Also found and read Doyle 1978 an earlier review of 1960s and 1970s research on ultrasonics


\subsection{Evolution of Available Sensors and Edge Computing techniques}
Sensor technology with MEMs accelerators and gyroscopes in the late 90s Edge computing and fog through the early 2000s


\pagebreak
\section{Chapter 3: Methods Proposed}
\subsection{Proposed Data Collection}

Preliminary data collection has been begun with a 3 axis acceleration sensor and 3 axis gyroscope connected to a Raspberry Pi.  The system is zip tied on a tray in the equipment house on an ASC.  This sensor is then left in position for several days collecting movement data.  Every few days the sensor needs to be physically removed, have the stored data copied off, and then installed on another crane, as seen in figure \ref{fig:mounted_pi}.  This has been collecting data since July 11, 2021.  The current system uses python to to read the acceleration as often as the loop runs, and save the data off after 60,000 reads.  The files for Acceleration and Gyrscopic Acceleration are saved and zipped for size storage.  Collecting 60,000 rows represents about 9 to 10 minutes of data.  A time stamp from beginning of the loop is stored with each record, and the file name is stored as the end time of file.  

\begin{figure}
	\centering
	\includegraphics[width=0.7\linewidth]{"Figures/Pi on Shelf"}
	\caption[Sensor on a Shelf in the ASC House]{The Raspberry Pi 3+ with a MPU 6050 3 axis gyroscope and 3 axis accelerator mounted in the case is installed on a shelf in the electrical house on the ASC.  The house is 6 feet off the ground, and this shelf is 7feet above the floor of the house.  The shelf has a UPS system should power fail.  The Pi is zip tied to a rigid electric conduit}
	\label{fig:mounted_pi}
\end{figure}

In the future there is potential to permanently mount a computer in the equipment house connected to acceleration sensors installed closer to the wheels and tracks.  The crane has 10GBs fiber connection to the port’s network that can be used to collect data, instead of the manual method currently.  The total generated would be too large for the network as a whole, even though a small percentage of each link.  Data could be stored on the crane for a rolling 14 days, and data only pulled remotely when there has been an event that is to be studied.

Additional vibration, ultrasound, or acceleration sensors as discussed above could be placed directly on the rails.  The end of the rail is exposed and above ground.  Potentially sensors can be mounted on the surface, the web, the end, and bottom of the rail looking for different signals.  Further analysis can be conducted with standard ultrasound equipment such as a rail analysis walking stick discussed above.

\subsection{Data Preparation}
Manually label acceleration curves for a number of the les. Label as Acceleration, Coasting, Decelerating, Idling. Then train a Neural Network to automatically segment each acceleration into the 4 types, properly labeling them. I will then be able to use that network to label all 3 months of data. Next step will be a xing any additional metadata that are available to those segmented sections. 

\subsection{Analysis Technique}
Standard wavelet families are made to analyze a whole signal, and work across portions of a signal.  Examples of Wavelet families are seen in figure \ref{fig:wavelet_families}.  

\begin{figure}
	\centering
	\includegraphics[width=0.7\linewidth]{"Figures/wavelet_families"}
	\caption[Wavelet Families]{Wavelets are made up of different families of functions that have similar form, and changing parameters.  Families are generally designed such that subsequent members of the family are orthogonal to each other, allowing a complete decomposition of a signal if they are all used.}
	\label{fig:wavelet_families}
\end{figure}
Wavelets are made up of a carrier signal and a changeable signal.  The Gaussian wavelets are a commonly used family that work by using a Gaussian or normal distribution as the Carrier signal, and then multiplying it by either a sine or cosine signal.  The sine or cosine signal is then scaled at different frequencies.  The combined scaled wavelet is correlated with the signal, and the correlation between the signal and the scaling wavelet is stored at each position in time for each scaling factor.

\begin{figure}
	\centering
	\includegraphics[width=0.7\linewidth]{"Figures/Gassian Wavelet with Carrier"}
	\caption[Gaussian Wavelet]{The Gaussian Wavelet uses the Gaussian Distribution, also known as the normal Distribution, as it's carrier signal.  The standard deviation $\sigma$ of the function is the primary feature of the distribution that is changed.  The mean is kept at 0 for all members of the family.  The carrier signal shown here is a simple sine wave.  When the two are multiplied together the Gaussian Distribution is created.}
	\label{fig:gaussian_wavelet}
\end{figure}

Through the research different wavelet families will be experimented with, and the optimum for the analysis will be used.  A wavelet that shows promise for live signal processing is the beta wavelet.  It uses the Beta Distribution as its carrier signal, which has two parameters that can be modified by study, and a scalable signal.  The Beta Wavelet has been shown to be useful in real time signal processing for ECGs.  

\begin{figure}
	\centering
	\includegraphics[width=0.7\linewidth]{"Figures/Beta Wavelet"}
	\caption[Beta Wavelet]{The Beta Wavelet carrier signal is a Beta Distribution.  In this example with $\alpha$ given by 2 and $\beta$ by 5.  The carrier is then multiplied by a sine wave, to give the Wavelet.  In this case, the Carrier signal uses the variable -x, allowing the processing of the historic signal only.}
	\label{fig:beta_wavelet}
\end{figure}

Wavelet based on single directional distribution. Will allow analysis of near recent history. Current wavelet needs to analyze a whole signal, this will allow the analysis of a live stream of data.  The Use of the beta function gives flexibility in changing the two beta parameters to determine which best works with the wavelet for feature selection.

\begin{equation}
Beta(x) = \frac{x^{1-\alpha}(1-x)^{1-\beta}}{B(\alpha,\beta)}
\end{equation}

Where B($\alpha$,$\beta$) is given by the equation
\begin{equation}
B(\alpha,\beta) = \frac{\Gamma(\alpha)\Gamma(\beta)}{\Gamma(\alpha+\beta)}
\end{equation}
Where $\Gamma$ is the gamma function.

This will then be used in conjunction with the Dynamic Wavelet Fingerprinting technique pioneered by Huo in 2004, with continued research implementing it for diverse problems from structural health monitoring in aircraft stringers, 5G signal analysis, and social media bot detection


\subsection{Comparative Analysis techniques}
Pit my new Wavelet against the standard Wavelet based approaches.  How does a live signal processing wavelet work against the full signal analysis standard wavelets in DWFP.  We will also run the direct signal into Recurrent Neural networks and attempt to train on all of the dimensional that is encompassed in the raw data.  There are several options for the type of RNN to use, either a Long Short Term Memory Model, or other types.  A third will compare feature selection out of the live Wavelet Fingerprint, RNNs trained on the resultant features. 

\pagebreak
\section{Chapter 4: Approach and Pilot Data}

Initially an acceleration sensor was coupled with a Raspberry Pi and placed in the House of an ASC. The sensor has then been moved in between multiple cranes. Over the course of the several months that the sensor has been collecting data, it has been on tracks before and after they were ground for smoothness, on tracks that were not scheduled for grinding, and on new tracks for the recently added cranes.  Shown in figure \ref{fig:Sample-Acceleration-Curve} is a sample of the data collected.  This is from a track that the on the ground maintenance technicians believe to be in poor condition.. 

\begin{figure}
	\centering
	\includegraphics[width=0.7\linewidth]{"Figures/Average Smoothed Acceleration Curve"}
	\caption[Sample Acceleration]{This shows the original and the smoothed (50 point averaging) acceleration in the dimension roughly parallel to the crane track. The black line is the 50 point average smoothed signal, and the gray dashed line is the raw signal.  The periods of acceleration, coasting, deceleration, and idling can be clearly seen in the averaged signal. The amount of noise in the raw signal makes it very hard to detect direction of change of travel, but the smoothing allows for clear visual interpretation for what is happening at each section of movement.  The signal is from a series of movements delivering containers to be loaded onto a ship, in a stack where the track was known by the maintenance technicians to be in need of grinding.}
	\label{fig:Sample-Acceleration-Curve}
\end{figure}

To clean the signal, a standard deviation of the last 50 points was taken at each recording, seen in figure \ref{fig:Smoothed_StdDev}.  The cranes are all set to move at a full speed of 5 meters per second.  A difference in standard deviation between the tracks will be good proxy for the roughness of the track, as felt at the sensor.  The signal when segmented for period of constant speed should be comparable between different cranes, and those with the higher standard deviation during movement are the ones likely in most needs of grinding.  Even on overall smooth tracks, if the locations of the movements are known, specific rough spots on the tracks can be isolated.

\begin{figure}
	\centering
	\includegraphics[width=0.7\linewidth]{"Figures/Smoothed and StdDev"}
	\caption[Smoothed and StdDev]{Showing the same signal as figure 2, with the 50 point average smoothed signal in black, and the standard deviation of the original signal over time in dashed gray.  Since the initial goal of the preliminary research and data collection phase is to highlight tracks in need of grinding, the standard deviation of the signal is a good proxy for the roughness of the track.  Since all the cranes move at the same velocity, the differences in the standard deviation of the acceleration while they are at constant speed should indicate the roughness of the track that is transferred to the sensor.}
	\label{fig:Smoothed_StdDev}
\end{figure}

Focussng on the area from 24,000 to 36,000 in the graph, there is a period of the crane waiting for its next move, the crane accelerating to full speed, coasting at full speed briefly, decelerating, and then waiting again for its next move.  

\begin{figure}
	\centering
	\includegraphics[width=0.7\linewidth]{"Figures/From24000to36000"}
	\caption[Smoothed and StdDev]{Dynamic Wavelet Fingerprint of the standard deviation curve of a sample crane move out of Figure 3.  The features of the fingerprint can be analyzed, both visually and mathematically, using techniques by Huo, Bingham and others.}
	\label{fig:StdDevFingerprint}
\end{figure}


%%%%%%%%%%%%%%%%%%%%%%%%%%%%%%%%%%%%%%%%%%%%%%%%%%%%%%%%%%%%%%%%%%%%%
%% The "Acknowledgement" section can be given in all manuscript
%% classes.  This should be given within the "acknowledgement"
%% environment, which will make the correct section or running title.
%%%%%%%%%%%%%%%%%%%%%%%%%%%%%%%%%%%%%%%%%%%%%%%%%%%%%%%%%%%%%%%%%%%%%
\pagebreak
\begin{acknowledgement}

The author thanks his wife for supporting him and not murdering him in his sleep for the insurance money.  His kids, Will, Liz, and Luke for being supportive, interested, and only distracting him sometimes with board games and minecraft.  Also would like to thank Wikipedia for actually getting people to contribute useful information.  He also thanks Mark Thorsen the CIO of the port for allowing him to install whatever software he wants on the computer without having to go back to helpdesk, allowing him to try 15 different LATEX editors before settling on this one, 4 different python IDEs, and 2 different version management systems.

\end{acknowledgement}

%%%%%%%%%%%%%%%%%%%%%%%%%%%%%%%%%%%%%%%%%%%%%%%%%%%%%%%%%%%%%%%%%%%%%
%% The same is true for Supporting Information, which should use the
%% suppinfo environment.
%%%%%%%%%%%%%%%%%%%%%%%%%%%%%%%%%%%%%%%%%%%%%%%%%%%%%%%%%%%%%%%%%%%%%
\begin{suppinfo}

The code used to generate these can be found on the author's GitHub page.

The following files are available free of charge.
\begin{itemize}
  \item GitHub Link: https://github.com/danchendrickson/Prospectus
\end{itemize}

\end{suppinfo}

%%%%%%%%%%%%%%%%%%%%%%%%%%%%%%%%%%%%%%%%%%%%%%%%%%%%%%%%%%%%%%%%%%%%%
%% The appropriate \bibliography command should be placed here.
%% Notice that the class file automatically sets \bibliographystyle
%% and also names the section correctly.
%%%%%%%%%%%%%%%%%%%%%%%%%%%%%%%%%%%%%%%%%%%%%%%%%%%%%%%%%%%%%%%%%%%%%
\bibliography{Prospectusbib}
	

%TC:endignore   

\end{document}
